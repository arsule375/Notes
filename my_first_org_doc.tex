% Created 2021-11-27 Sat 19:55
% Intended LaTeX compiler: pdflatex
\documentclass[11pt]{article}
\usepackage[utf8]{inputenc}
\usepackage[T1]{fontenc}
\usepackage{graphicx}
\usepackage{longtable}
\usepackage{wrapfig}
\usepackage{rotating}
\usepackage[normalem]{ulem}
\usepackage{amsmath}
\usepackage{amssymb}
\usepackage{capt-of}
\usepackage{hyperref}
\usepackage{minted}
\author{Atharv Sule}
\date{\today}
\title{Demo2}
\hypersetup{
 pdfauthor={Atharv Sule},
 pdftitle={Demo2},
 pdfkeywords={},
 pdfsubject={},
 pdfcreator={Emacs 27.2 (Org mode 9.6)}, 
 pdflang={English}}
\begin{document}

\maketitle

\section{This is a  Toplevel Heading}
\label{sec:org2feb7b4}

Here's: \textbf{bold} \emph{italics} \uline{underline} \texttt{code}.

\subsection{Subheading}
\label{sec:org1b3f0cf}

Numbered list:

\begin{enumerate}
\item First item
\item Second item
\begin{enumerate}
\item Sublist
\item Sublist item 2
\begin{enumerate}
\item Subsublist
\end{enumerate}
\end{enumerate}
\item Some more items
\end{enumerate}

Bulleted lists:
\begin{itemize}
\item Bullet 1
\item Bullet 2
\begin{itemize}
\item Subbullet
\item Etc.
\end{itemize}
\end{itemize}

Some text here

\newpage

\section{Another Toplevel Heading}
\label{sec:orgbb2ab5b}
Lets do some more advanced stuff

\subsection{Code Blocks}
\label{sec:org4c58c7a}

\begin{minted}[]{python}
print"Hello world"
\end{minted}

\begin{verbatim}
Hello world
\end{verbatim}




Heres's some Java:

To execute a code block do \texttt{C-c C-c}

\begin{minted}[]{java}
// your first program
class Helloworld {
    public static void main(String[] args){
        System.out.println("Hello There!");
    }
}
\end{minted}

\begin{verbatim}
Hello There!
\end{verbatim}



\begin{minted}[]{cpp}
#include <stdio.h>

int main(int argc, char *argv[]){
  printf("Hello world");
  return 0;
}
\end{minted}

\begin{verbatim}
Hello world
\end{verbatim}



\begin{minted}[]{common-lisp}
(princ "Hello world ")
\end{minted}

\subsection{Tables}
\label{sec:org83749d3}

How to make table type | name | name | <tab> \texttt{C-c} <enter

\begin{center}
\begin{tabular}{ll}
name & Ice Cream\\
\hline
Atharva & vanilla\\
adi & Vanilla\\
Rahul & Stawberry\\
Vijay & none\\
\end{tabular}
\end{center}


\begin{center}
\begin{tabular}{ll}
name & Car\\
\hline
Atharv & ferrari\\
adi & laborgini\\
dad & honda\\
\end{tabular}
\end{center}

\begin{center}
\begin{tabular}{ll}
name & Home\\
\hline
Atharv & US\\
\end{tabular}
\end{center}


\section{Basic Emacs}
\label{sec:org6f7a6fb}
\begin{itemize}
\item \texttt{C-x C-f}: open a file (old or new doesn't matter)
\item \texttt{C-x C-c}: exit Emacs (will ask for confimation, type ``y'')
\item \texttt{C-x C-s}: save file
\item \texttt{M-x org-latex-export-to-pdf}: generate PDF output for Org file.
\item \texttt{M-w} copy \texttt{C-y} paste.
\item \texttt{C-c C-l} Hyperlinks.
\end{itemize}

\section{Mathematics}
\label{sec:org3f066e5}

Hyperlinks:\href{https://nebhrajani-a.github.io/}{adi's website hi}

Math equation:
Example:
$$\sum_{i=0}^n i^2 = \frac{(n^2+n)(2n+1)}{6}$$

\section{Org Mode Manuel}
\label{sec:org2b34287}
\href{https://orgmode.org/manual/}{Click here}
\end{document}
