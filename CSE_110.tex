% Created 2021-12-02 Thu 23:18
% Intended LaTeX compiler: pdflatex
\documentclass[11pt]{article}
\usepackage[utf8]{inputenc}
\usepackage[T1]{fontenc}
\usepackage{graphicx}
\usepackage{grffile}
\usepackage{longtable}
\usepackage{wrapfig}
\usepackage{rotating}
\usepackage[normalem]{ulem}
\usepackage{amsmath}
\usepackage{textcomp}
\usepackage{amssymb}
\usepackage{capt-of}
\usepackage{hyperref}
\usepackage{minted}
\author{Atharv Sule}
\date{\today}
\title{Notes of CSE\textsubscript{110} Fall 2021}
\hypersetup{
 pdfauthor={Atharv Sule},
 pdftitle={Notes of CSE\textsubscript{110} Fall 2021},
 pdfkeywords={},
 pdfsubject={},
 pdfcreator={Emacs 27.2 (Org mode 9.4.4)}, 
 pdflang={English}}
\begin{document}

\maketitle
\tableofcontents


\section{Auditorium suorce code}
\label{sec:org7a541cc}


\begin{minted}[]{java}

import java.io.File;
import java.io.FileNotFoundException;
import java.io.PrintWriter;
import java.util.Scanner;
import java.text.NumberFormat;

public class Auditorium {
    double[][] seats;
    double totalSales;
    int numSold;

    // default constructor
    public Auditorium ()  {
	seats = new double[3][4];
	// to view path with file in pathname and click on the file
	try {
	    File inputFile = new File("seatPrices.txt");
	    Scanner in = new Scanner(inputFile);
	    while (in.hasNextDouble()){
		for(int i = 0; i < 3; i++ ){
		    for(int j = 0; j < 4; j++){
			double value = in.nextDouble();
			seats[i][j] = value;
		   }
	       }
	    }
	} catch (FileNotFoundException e) {
	    e.printStackTrace();
	}
	System.out.println();
       totalSales = 0;
       numSold  = 0;
    }

    // gets the total price of the tickets sold
    public String getTotal(){

	NumberFormat fmt = NumberFormat.getCurrencyInstance();
	fmt.format(totalSales);
	return "" +  fmt.format( totalSales);
    }

    public void displayChart(){
	for(int i = 0; i < 3; i++ ){
	    for(int j = 0; j < 4; j++){
		System.out.print( seats[i][j] + "    ");;
	    }
	    System.out.println("");
	}
    }

    // used to sell tickets by setting ticket value to zero
    public boolean sellTicket(int i, int j){
	for(int l = 0; l < 3; l++ ){
	    for (int m = 0 ; m < 4; m++){
		if (( i == l) && (j == m)){
		    if(seats[l][m] != 0){
			totalSales = totalSales + seats[l][m];
			numSold++;
			seats[l][m] = 0.0;
			return true;
		    }
		}
	    }
	}
	return false;
    }

    // gets number of tickets sold
    public int numSold(){
	return numSold;
    }

    // checks if tickets are sold out or not
    public boolean soldOut(){
	for(int l = 0; l < 3; l++ ){
	    for (int m = 0 ; m < 4; m++){
	       if(seats[l][m] != 0){
		   return false;
	       }
	    }
	}
	return true;
    }
}


\end{minted}

\section{Sorting Algo's}
\label{sec:org971f8c6}

\begin{minted}[]{java}

import java.io.*;
import java.util.*;

public class Sorting {
    public static void main(String[] args) {
	int arr[ ] =  { 8, 6, 9, 3 ,4, 5 };

		// //  // selection sort
       // selectionSort(arr);

	System.out.println("");
	// //  // insertion sort
	String stringArr2 = Arrays.toString(insertionSort(arr));
	System.out.println(stringArr2);


	System.out.print(bianarySerch(insertionSort(arr), 4));

	System.out.println("");

    }

    private static int[] insertionSort(int[] arr) {
	for (int i = 1; i < arr.length; i++){
	    int j  = i;
	    while (j > 0 && (arr[j -1] > arr[j])){
		int tem = arr[j];
		arr[j] = arr[j - 1];
		arr[j - 1] = tem;
		j--;
	    }

	    String array2 = Arrays.toString(arr);
	    System.out.println(array2);

	}


	return arr;
    }

    private static void selectionSort(int[] arr) {
	for(int j = 0; j < arr.length; j++ ){
	    int min = j;
	    for (int i = j + 1; i < arr.length; i++){
		if (arr[i] < arr[min]){
		    min = i;
		}
	    }
	    if (min != j){
		int temp = arr[j];
		arr[j] = arr[min];
		arr[min] = temp;
	    }
	    String array1 = Arrays.toString(arr);
	    System.out.println("Phase" + (j + 1) + ":" +  array1);
	}

	for (int m = 0; m < arr.length; m++){
	    System.out.print(arr[m] + " ");
	}
    }

    private static int  bianarySerch(int[] arr, int num){
	int left = 0;
	int right = arr.length - 1;
	while (left <= right ){
	    int mid = (left + right)/2;
	    if (arr[mid] == num){
		return mid;
	    }else if (num < arr[mid]){
		right = mid - 1;
	    }else{
		left = mid + 1;
	    }
	}

	return -1;
    }
}


\end{minted}

\begin{verbatim}

[6, 8, 9, 3, 4, 5]
[6, 8, 9, 3, 4, 5]
[3, 6, 8, 9, 4, 5]
[3, 4, 6, 8, 9, 5]
[3, 4, 5, 6, 8, 9]
[3, 4, 5, 6, 8, 9]
[3, 4, 5, 6, 8, 9]
[3, 4, 5, 6, 8, 9]
[3, 4, 5, 6, 8, 9]
[3, 4, 5, 6, 8, 9]
[3, 4, 5, 6, 8, 9]
1
\end{verbatim}
\end{document}
