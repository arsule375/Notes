% Created 2021-12-27 Mon 22:12
% Intended LaTeX compiler: pdflatex
\documentclass[11pt]{article}
\usepackage[utf8]{inputenc}
\usepackage[T1]{fontenc}
\usepackage{graphicx}
\usepackage{grffile}
\usepackage{longtable}
\usepackage{wrapfig}
\usepackage{rotating}
\usepackage[normalem]{ulem}
\usepackage{amsmath}
\usepackage{textcomp}
\usepackage{amssymb}
\usepackage{capt-of}
\usepackage{hyperref}
\usepackage{minted}
\author{Atharv Sule}
\date{\today}
\title{Investing lecture series}
\hypersetup{
 pdfauthor={Atharv Sule},
 pdftitle={Investing lecture series},
 pdfkeywords={},
 pdfsubject={},
 pdfcreator={Emacs 27.2 (Org mode 9.4.4)}, 
 pdflang={English}}
\begin{document}

\maketitle
\section{}
\label{sec:org5406e4a}
\section{}
\label{sec:orge40a6bb}
\section{Part 3: How to think about Investing}
\label{sec:org005002c}
\subsection{What foundations do we need before investing our hard-earned money?}
\label{sec:orgac7d639}
\begin{itemize}
\item Try not to get yourelf into  any of the following dept:
\begin{itemize}
\item Morage dept( you must be able to afford morgage)
\item Student Dept( Try to aviod if you can, most curcumstantual)
\item Credit dept ( dont spend reclessly)
\end{itemize}
\item Set up an emergancy fund
\begin{itemize}
\item Most people live paycheck to paycheck, but you must prepare of the scenaro of no paycheck
\item money can cover your expences in case of emergancy
\end{itemize}
\end{itemize}
\subsection{Should we worry about inflation or deflation?}
\label{sec:orgc6d3990}
\begin{itemize}
\item Why not keep exess cash under your bed?
\begin{itemize}
\item Inflation causes the value of money to derease over time
\item Ex: The price of a tall mocha went up 2.4x in 16 years
\item Ex: Rent for 1 bedroom apartment is up 2.8x in 16 years
\item The examples above are examples of inflation
\item The moneys purchasing power got reduced
\end{itemize}
\item Technology is an example of defaltion
\begin{itemize}
\item EX:If you wait for some time the value of a tech product goes down
\item Ipones now are checper and more powerful than computers in the 1920
\end{itemize}
\item You must balance inflation and deflation to have a stable economy
too much of etheir can lead to economic collpase.
\begin{itemize}
\item Infleation : price going up promps the users to buy goods at the time they can
afford them
\item Deflation: no none wants to buy and price contiuously goues down
\end{itemize}
\item Inflation is increaseing especially in the US, so its good to invest as
soon as possible
\end{itemize}
\subsection{What kind of assets do US households own?}
\label{sec:org6dace77}
\begin{itemize}
\item Finantial assets: backed by a company, derives value from contractual claim(EX: stocks, cash, funds)
NonFinatial assets: not backed by company, derives value from phyical state
\item American households invest mostly in equities and corporal fund shares
\item with assets, if safety is high the duration, return and volatity is low
\item when companies fail finitailly shareholders get hit the hardest
\item shareholders vs bondholders:
\item You get more intrest out of unsafe fintial assets
\end{itemize}
\subsection{what kind of assets do US hoeholds have}
\label{sec:orgf011ae9}
\begin{itemize}
\item The best investment to make is one in yourself
\end{itemize}


\section{}
\label{sec:org1cd6a04}
\section{}
\label{sec:orgbf93487}
\end{document}
